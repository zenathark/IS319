\section{Memory Management} 
Una de las actividades principales del kernel es el manejo de memoria RAM. En
lo subsecuente se hará referencia a la memoria RAM solamente como memoria del
sistema. Como puede deducirse después de estudiar el arranque de una PC, el
sistema operativo debe de proveer una capa de abstracción para el acceso a
memoria. La opción mas sencilla es dejar que cada aplicación tenga acceso
directo a la memoria y la maneje de manera independiente al resto de las
aplicaciones. Sin embargo esto nos enfrenta a diferentes problemáticas. En
principio la memoria tendría que manejarse a bajo nivel usando direcciones
físicas. El manejo de direcciones físicas dificultaría el desarrollo de
aplicaciones. Además abriría la posibilidad de que una aplicación tenga acceso
al espacio de memoria de otra aplicación y sobreescribirla o peor aun,
sobreescribir el mismo kernel o los servicios del BIOS.  Por esta razón se opta
en el diseño de kernels por crear una capa de abstracción de la memoria
apuntando a incluir las siguientes características:
\begin{enumerate}
\item Seguridad
\item Transparencia
\item Eficiencia en el manejo del espacio disponible
\item Manejo de grandes volúmenes de aplicaciones activas
\item Flexibilidad
\end{enumerate} Estos objetivos intentan dotar al sistema operativo con una
capa de abstracción de manejo de memoria que le permita ofrecer un ambiente
confiable y eficiente a cada aplicación además de ser lo suficientemente
flexible para ejecutar aplicaciones que requieran más memoria de la que
físicamente se encuentra disponible. Para poder explicar como se resuelven
todos estos puntos, primero se describirá como se encuentra direccionada la
memoria física.

\subsection{Memory Addressing} 
La arquitectura actual de procesadores usa un modelos de memoria física
conocido como modelo Von Neuman. En este modelo se tiene un solo espacio de
memoria que es compartido tanto por el código ejecutable como por los
daros. Existen diferentes métodos para manejar la memoria física que son
soportados de manera nativa para contribuir con la creación de una capa de
abstración para el manejo de memoria por parte del sistema operativo. El
controlador de memoria o Memory Management Unit (MMU)de la arquitectura x86
soporta dos modos de operación: \emph{Real Mode} y \emph{Protected Mode}.  Real
Mode es un modo de operación con acceso directo a las direcciones físicas, este
modo se incluye para mantener una compatiblidad hacia atrás con los antiguos
sistemas operativos y para el arranque usando un bootloader. El modo protegido,
que es en el que estaremos trabajando, se describe a continuación.  En
Protected Mode se utilizan tres tipos de direcciones:
\begin{enumerate}
\item Logical Address. Estas direcciones son las manejadas a nivel de lenguajes
  de bajo nivel como Assembler y se forman de un \emph{segment} y un
  \emph{offset}. Un segment es la direccion inicial del bloque de memoria
  asignado a un proceso\footnote{A partir de ahora se hará referencia a
    procesos en lugar de aplicaciones, en el siguiente capítulo se aclarará
    sistemáticamente la diferencia entre ambos.} y un offset es el
  desplazamiento a partir de la dirección inicial contenida en
  segment. Usualmente solo se debe tener conciencia de estas direcciones cuando
  se manejan lenguajes o librerías de bajo o medio nivel como C. Estas son las
  únicas direcciones a las que tenemos acceso en Protected Mode.
\item Linear Address. Este tipo de direcciones etiquetan las zonas de memoria
  usando un único índice de 32bits o 64bits dependiendo el tipo de
  arquitectura. 32 bits pueden diccionar hasta 4GB de memoria mientras que 64
  bits pueden direccionar 2B aproximadamente. Esta es la razón por la que
  sistemas operativos de 32 bytes tienen que usar una modificación especial
  cuando requieren direccionar memoria mas allá de los 4GB.
\item Physical Address. Esta es la verdadera dirección de una celda de
  memoria. Esta dirección hace referencia a la señal eléctrica de los pines de
  la memoria y se representa con un \texttt{unsigned int}. De 32 bits o 36 bits
  para procesadores de 32 bits y 64 bits o para procesadores de 64 bits.
\end{enumerate} El acceso a memoria a nivel de hardware es manejado por el
\emph{Memory Arbiter}. Esta pieza de hardware permite a cada CPU presente
acceder a la memoria así como también a los diferentes dispositivos usando
\emph{Direct Memory Access} sin necesidad de solicitarlo al CPU. Cada CPU
requiere un Memory Arbiter dedicado.  La physical address es obtenida a nivel
de hardware

\subsection{Segmentation}

Segmentation fue el primer método introducido para incorporar mecanismos de
memoria virtual y seguridad en el manejo de memoria. Este mecanismo usa
direcciones llamadas Logical Address. En la arquitectura Intel, segmentation es
soportado desde el modelo 80286. Para poder entender la manera de calcularlo es
necesario revisar la estructura binaria de las Logical Address.  El primer paso
es obtener la linear address utilizando la logical address. El hardware de la
arquitectura x86 calcula de diferente manera la linear address en sus
diferentes modos de operación. Para protected mode, el hardware utiliza una
estructura de 16 bits para el segment o \emph{segment selector} y 32 bits para
el offset en procesadores de 32 bits. Mientras que el offset es un unsigned int
que informa el desplazamiento desde la dirección inicial asignada, el campo de
segment tiene una estructura más compleja. La estructura se muestra en la
figura .

Los bits están enumerados de derecha a izquierda a modo \emph{little indian}
como es el estandar de la arquitectura x86, siendo el bit menos significativo o
LSB por las sigla en inglés de \emph{Less Significant Bit} el bit de más a la
derecha con índice \texttt{0} y el bit más significativo o MSB por las siglas
en ingles de \emph{Most Significant Bit} con índice 15 en el caso del segment
selector. Los 13 bits mas significativos son llamados \emph{index} y mantiene
el índice de una entrada slmacenada en una \emph{Table Descriptor} explicada
más adelante. El bit 2 indica el tipo de Table Descriptor a la que apunta y por
úiltimo los bits 0 y 1 indican el nivel de permiso que tiene el segmento. linux
en particular usa solo el valor 0 para kernel mode y 3 para user mode. A cada
proceso creado se le otorgan tres segmentos: el code segment donde reside el
código ejecutable, el stack segment donde se almacenan valores dinamicamente a
manera de pila y el data segment donde se almacenan valores a manera de acceso
indisado. La diferencia entre el data segment y el stack segment reside en la
manera de acceder a los datos.  El procesador mantiene los valores de los
distintos segment en los llamados registros de segmento. Son seis registros de
segmento, tres de uso particular y tres de uso general. Los tres registros de
uso particular son: \texttt{ds} para el data segment, \texttt{ss} para el stack
segment y \texttt{ds} para el data segment. Los tres registros de propósito
general son: \texttt{es}, \texttt{fs} y \texttt{gs}.  Para poder traducir
rapidamente de la logical address a la linear address, el sistema operativo
utiliza un campo de 8 bytes llamado \emph{segment descriptor}. Los segment
descriptor son almacenados en tablas. Las tablas que almacenan los segment
descriptor pueden ser de tipo \emph{Global Descriptor Table} (GDT) o de tipo
\emph{Local Descriptor Table} (LDT). Usualmente solo se mantiene una GDT por
cada procesador presente y su tamaño y dirección son almacenados en el registro
\emph{gdtr}. Las LDT son creadas por los procesos que requieren de reservar mas
memoria y su tamaño y dirección son almacenados en el registro
\texttt{ldtr}. Cada proceso tiene una sola tabla LDT. Existen varios tipos de
segment descriptor pero el hardware mantiene solo tres estructuras físicas. La
estructura física para los procesadores de 32 bits se describe en la figura {}.

La tabla \ref{table:segmentd} describe los diferentes campos de los segment descriptors.
\begin{table*}[ht]
\begin{center}
\begin{tabular}{p{1cm}p{.8\linewidth}} 
  \multicolumn{1}{c}{Nombre} & \multicolumn{1}{c}{Descripción}\\
  \hline
  Base   & Contiene la linear address del primer byte del segment\\ 
  G      & Granularity Flag. Si esta en 0 el tamaño se expresa en bytes. Si esta
           en 1 el tamaño se expresa en múltiplos de 4096 bytes\\ 
  Limit  & Contiene el desplazamiento al último byte del segment indicando asi
           también el tamaño del segmento. Si G tiene el valor de 0 el tamaño va 
           de 1 byte hasta 1 MB. Si G es 1 el tamaño va de 4 KB hasta 4 GB. \\ 
  S      & System Flag. Si es 0,  el segment descriptor es un system segment 
           descriptor. Si es 1 es un data o code segment descriptor\\ 
  Type   & El tipo de segment descriptor y sus permisos\\ 
  DPL    & Tipo de privilegio necesario para acceder a este segment descriptor. 
           Si es 0 solo puede accedre el kernel (kernel mode). Si es 3 cualquier 
           proceso puede acceder al segment descriptor (user mode)\\ 
  P      & Present Memory Flag. Indica si: 0, el segmento no está en la memoria 
           o 1 si se encuentra. Este concepto esta relacionado con el swapping 
           entre la memoria y los dispositivos de almacenamiento. El swapping no 
           necesariamente ocurre con segments completos, por lo que sistemas 
           como Linux siempre ponen el valor de 1 ya que nunca hace swapping a 
           todo el segment. \\
  D or B & indica si el offset del segment es de 32 bits cuando esta en 1 o de 
           16 bits cuando esta en 0 \\ 
  AVL    & Bandera de propósito general para el sistema operativo. 
\end{tabular}
\end{center}
\caption{Flags del Segment Descriptor}
\label{table:segmentd}
\end{table*}

Como se ha mencionado, existen diferentes tipos de segment descriptor. Aunque
físicamente se usa la misma estructura, las banderas indican el tipo de
descriptor que se esta almacenando. Este tipo de prácticas solo es válido
cuando se trabaja a este nivel de abstracción. En lenguajes de mas alto nivel
se deberían crear estructuras de datos con diferentes nombres. La manera de
manejar los segment descriptor quedan a discreción del sistema operativo. En eñ
caso de linux se manejan cuatro tipos:

\begin{itemize}
\item Code Segment Descriptor. Este describe un code segment. La bandera S esta
  encendida y la bandera Type almacena el valor que representa un code segment
  descriptor. Las otras banderas dependen del proceso al que pertenece. El
  segment puede encotrarse en una GDT o una LDT.
\item Data Segment Descriptor. Este describe un stack o data segment. La
  bandera S esta encendida y la bandera Type almacena el valor que representa
  un data segment descriptor. Para linux, el data y el stack segment no
  necesitan diferenciarse para calcular su linear address, debido a esto solo
  se usa un tipo de segment descriptor. El segment puede encontrarse en una GDT
  o una LDT.
\item Task Segment Descriptor. Describe el estado de un proceso. En linux un
  proceso se suele llamar Task o Thread. Este solo se puede encontrar en la
  GDT. El campo Type contiene 11 o 9 dependiendo de si el proceso se encuentrs
  en el CPU o no respectivamente. La bandera S tiene el valor de 0.
\item Local Table Table Descriptor. Este descriptor indica que el segmento
  apunta a una LDT. La bandera S esta apagada y Type tiene el valor de 0. Solo
  se puede almacenar este tipo en una GDT.
\end{itemize}

El procesador x86 mantiene registros no programables por cada segment register
que almacena los correspondientes segment descriptors. El CPU puede usar estos
registros no programables para traducir las logical address en linear
address. Estos registros no programables se escriben automáticamente al momento
de que un proceso entra en ejecución. Cuando el CPU se ve en la necesidad de
calcular la linear address de una logical address que no esta en ejecución
utiliza el algoritmo mostrado en la tabla {}.

Los pasos del algoritmo son los siguientes:
\begin{enumerate}
\item El CPU revisa el campo TL del segment selector para determinar si el
  descriptor esta en la GDT o en la LDT. La dirección de la tabla la obtiene de
  los registros gdtr o ldtr. Note que el registro ldtr se tiene que reescribir
  cada que un proceso entra en ejecución. El gdtr se mantiene con el mismo
  valor desde que el kernel se carga en memoria.
\item El índice del segment descriptor se obtiene multiplicando el valor
  almacenado en la bandera index del segment selector por 8 que es el tamaño en
  bytes del segment descriptor y sumándole la dirección de la tabla obtenida en
  el paso anterior.
\item por último, la linear address se obtiene de sumar el offset de la logical
  address con la linear address almacenada en el segment descriptor.
\end{enumerate} 

En el caso de que la logical address pertenezca a un proceso en
ejecución se omiten el primer y segundo paso al tomar el segment descriptor de
los registros no programables. La tabla {} muestra un diagrama de la memoria
indicando a donde apuntan las distintas direcciones. Estas operaciones sirven
debido a que asumimos que el direccionamiento de memoria es linear y empieza en
0.

En general, los sistemas operativos actuales no hacen uso exteso de este
mecanismo debido a que otras arquitecturas diferentes a la x86 no hacen uso de
la segmentación. Esta decisión simplifica el trabajo de transportar el sistema
operativo a otros dispositivos con arquitecturas diferentes como es el caso de
los dispositivos móviles con arquitectura ARM que no soporta segmentación.

\subsection{Paging} 
Paging es un segundo método para crear un modelo virtual de memoria. Este
permite que cada segment asignado a un programa no necesite estar en
direcciones físicas contiguas como es el caso de usar segmentation. Para
matener la eficiencia en el direccionamiento de memoria, el kernel agrupa
múltiples linear address en una unidad llamada \emph{Page}. Todas las linear
address dentro de una page apuntan a direcciones físicas contiguas. Esto
simplifica el uso de permisos ya que solo se necesita llevar un registro del
tipo de permisos por Page en lugar de por linear address. Este mecanismo es
llevado a cabo a nivel de hardware por la \emph{Paging Unit}. La paging unit
divide a la memoria RAM en bloques llamados \emph{Page frames}. El tamaño del
page frame es el mismo de una page, lo que permite al CPU colcar una page en
cada page frame solamente. La diferencia entre una page y un page frame es que
una page es un grupo de linear address y los datos almacenados en cada linear
address y un page frame es un espacio de almacenamiento en memoria RAM. Una
page puede ser almacenada en otro lugar que no sea la memoria RAM.

La arquitectura x86 utiliza dos métodos para implementar paginación. El primer
método, o Paging, divide a la linear address en tres bloques. El primer bloque
de los diez bits mas significativos, es decir el bit 31 al 22, es llamado Page
Directory. Los siguientes diez bits son llamados Page Table y los últimos doce
bits son llamados offset. La intención de usar dos niveles es de reducir el uso
de memoria por tabla. Este esquema permite mantener una tabla de $2^10$ o 1 MB
para almacenar los Table Directory y cada Table Index es cargada a memoria
cuando es requerida por un proceso. Tanto la Table Directory como la Table
Index tienen la misma estructura. Este método permite Pages de 4KB de
tamaño\footnote{el offset usa 12 bits, $2^12=4096$}. Esta estructura es
mostrada en la tabla \ref{table:tabled}.

Las funciones de las banderas se explican a continuación. Los nuevos conceptos
que aparacen en la tabla se iran explicando a lo largo del resto del capítulo.
\begin{table*}[ht]
\begin{center}
\begin{tabular}{p{1cm}p{.8\linewidth}} 
  \multicolumn{1}{c}{Nombre} & \multicolumn{1}{c}{Descripción}\\
  \hline
  Present          & Tiene el valors 1 si la Page se encuentra en memoria. Si 
                     es 0 el resto de los campos no tienes significado. En caso
                     que un proceso solicite esta page la paging unit levanta 
                     una excepción llamada page fault. \\ 
  Physical Address & Los siguientes 20 bits representan las physical address en 
                     una Page Table. Solo se necesitan 20 bits en lugar de 32 
                     porque las pages son de 4 KB de tamaño. Por lo tanto la 
                     dirección inicial de una page es un múltiplo de 4096 y los 
                     12 bits menos significativos estarán apagados. En una Page 
                     Directory apunta a la Page Table que contiene la physical 
                     address.\\
  Accessed         & Toma el valor de 1 cuando la paging unit la usa para 
                     calcular una dirección solicitada. Este campo es usando 
                     junto con Present para implementar swapping. \\ 
  Dirty            & Toma el valor de 1 cuando alguna localidad de una Page 
                     Table es escrita. También se usa para swapping\\
  Read/Write       & Permisos de la page table, puede ser Read o Read/Write \\ 
  User/Supervisor  & Indíca el nivel de permiso requerido para pode acceder a 
                     la page.\\ 
  PCD y PWT        & Estas dos banderas indican como debe ser administrada la 
                     page cuando esta en memoria cache. \\
  Page size        & Indica el método de paginación \\ 
  TLB              & Esta bandera también se usa para el manejo de cache. 
\end{tabular}
\end{center}
\caption{Flags del Table Directory}
\label{table:tabled}
\end{table*}

El concepto de swapping mencionado en las banderas Present, Dirty y Accessed es
parte del me anismo de memoria virtual por el que fue concebido el Paging. Sin
olvidar la seguridad. Swapping es el mecanismo que intercambia una page
contenida dentro de un page frame a un dispositivo de almacenmiento secundario
y pone ptra page en su lugar. Esto se utiliza para extender el espacio
disponible en memoria en ciertos casos. El sistema de memoria virtual se
concibio debido a que, desde la creación de las computadoras, la memoria RAM
siempre ha sido mas cara que los dispositivos de almacenamiento magnético. El
sistema de memoria virtual nos permite extender la memoria disponible para
procesos que requieran una cantidad inferior a la memoria física
existente. Este método no sirve si se requiere de un bloque contiguo que exceda
el tamaño de la memoria física. La desventaja de este método es el que el
acceso a dispositivos de almacenamiento es varias veces mas lento que a
memoria. Esto provoca que un sistema que haga mucho swapping vea su rendimiento
reducido ya que involucra al menos tres acciones: solicitud de una page,
excepción de page fault por la paging unit, swapping por el sistema
operativo. El sistema operativo se encarga de decidir que page intercambia con
una en el dispositivo de almacenamiento al ver las banderas Dirty y
Accessed. El sistema operativo usa la page es la menos usada como candidata a
swapping.

El segundo método de paging se conoce como \emph{extended paging}. Este método
se incorporó en la arquitectura x86 a partir del modelo Pentium. En extended
paging se usan pages de 4 MB en lugar de 4 KB. Esto se logra dividiendo la
logical address en dos niveles en lugar de tres. Los primeros 10 bits mas
significativos son el Page Directory y los siguientes 22 son el offset de la
physical address. Los bits de los campos del Page Directory siguen siendo los
mismos excepto que el campo de la physical address solo se toman los primeros
10 bits mas significativos debido a que estamos manejando múltiplos de 4 MB y
los 22 bits menos significativos de estos múltiplos siempre se encuentran
apagados. Cual método usar depende de la aplicación. El método de Paging
decrece en rendimiento cuando un proceso solicita bloques de memoria
grandes. Esto es si el proceso accede constantemente a las diferentes pages que
componen el bloque de memoria que solicitó, lo que puede provocar múltiples
excepciones de page fault. Extended Paging presenta un mejor rendimiento en
estos casos al reducir el número de excepciones para este tipo de procesos. En
otros casos Paging presenta un mejor rendimiento promedio. Extended paging
tambien presenta una ventaja en el uso de cache explicado mas adelante.

Un problema que se presentó previo a la aparición de procesadores de 64 bits es
cuando se requeria direccionar mas de 4 GB de memoria física ya que solo se
manejaba una dirección de 32 bits\footnote{$2^{32}\simeq 4$ GB}. Este problema
aparece primero en servidores y mas adelante en equipos domésticos con el
abaratamiento de la memoria RAM. Para solucionar esto, Intel introduce el
mecanismo llamado Physical Address Extension (PAE) Paging a partir de los
procesadores Pentium Pro. Para esto, aumenta el número de pines de la memoria
de 32 a 36 e introduce un nuevo nivel de paging creando una tabla especial con
entradas de 64 bits que puedan almacenar los 36 bits necesarios. Note que se
usan 64 bits en lugar de los 36 ya que los procesadores solo manejan
localidades con tamaño múltiplo del tamaño el bus, es decir 32 bits en este
caso.

Para procesadores de 64 bits se mantuvo el tamaño de las pages en 4KB excepto
en ciertas arquitecturas como la \emph{alpha} de o la \emph{ia64} de Intel para
servidores donde el tamaño es variable. Para mantener la eficiencia en estas
arquitecturas se decidió introducir mas niveles de paging. Para la versión de
64 bits de la arquitectura x86 (x86\_64) se manejan 4 niveles de paging,
separando la linear address en cuatro bloques de pages de 9 bits cada uno y un
offset de 12 bits (4 KB). La cantidad de niveles varia de arquitectura en
arquitectura, un ejemplo de esto es la arquitectura PowerPC de IBM usada en las
antiguas computadoras apple y la Playstation 3 que usan tres niveles de
paginación separando la linear address en dos bloques de pages de 10 bits, un
bloque de page de 9 bits y un offset de 12 bits. Note que ninguna de las
arquitecturas mencionadas usan 64 bits para la linear address. Por el momento,
se consideró que 46 bits en la arquitectura x86\_64 pueden satisfacer la demanda
en direccionamiento de memoria. Esto equivale a direccionar $2^{46}$ direcciones
físicas lo que es al rededor de.

\subsection{Caching}
La velocidad de la memoria es hasta ahora menor que la de los procesadores.
Esto implica que el procesador queda esperando a la memoria por la
transferencia de una localidad. La diferencia de velocidades crea un cuello de
botella. Para poder reducir este efecto, se creó la memoria cache. Este tipo de
memoria es mas rápida debido a que es de tipo Static RAM en lugar de la usada
comunmente denominada Dynamic RAM. La cache se coloca dentro del dado el
procesador dandole mayor ventaja al estar mas cerca del CPU. La implementación
de este tipo de memoria usa el llamado \emph{Principio de Localidad}. Este
principio hace referencia a que despues de acceder a una localidad de memoria,
es altamente probable que se enseguida se acceda a alguna localidad
cercana. Por esto, la cache esta formada por varias unidades llamadas
\emph{lines}. Cada line puede almacenar varias localidades de memoria. La
transferencia entre la memoria principal y la cache sucede en rafagas donde se
transmiten varias localidades juntas. El modo de operación es similar al de
Paging. La cache es admininstrada por el cache controller. El cache controller
almacena un conjunto de bits llamado \emph{Tag} para identificar las
localidades de memoria de cada line. Cuando el procesador requiere una
localidad, el cache controller revisa el tag asociado y lo compara con los tag
de cada line. Si lo encuentra se tiene un \emph{cache hit}, en caso contrario
se tiene un \emph{cache miss}. Cuando ocurre un cache hit existen dos
comportamientos dependiendo de la operación a realizar. Cuando es de lectura,
el cache controller transfiere la localidad al CPU sin acceder a la memoria
RAM. En el caso de que la operación sea de escritura existen dos estrategias:
\emph{write through} y \emph{write back}. En write through, el cache controller
escribe la line tanto en la cache como en la memoria principal. En write back,
el cache controller solo escribe la line en la cache, esperando hasta que
exista una solicitud de \emph{flush} por parte del CPU para escribir las lines
modificadas a RAM. La solicitud \emph{flush} ocurre principalmente en un evento
de cache miss.

Cuando ocurre un cache miss, se realiza un flush de ser necesario y se
sobrescribe una linea por la solicitada por el CPU. En equipos con multiples
nucleos con cache propia y memoria RAM compartida, el cache controller siempre
tiene que revisar en una operación de escritura las cache de los otros
procesadores para actualizarla en caso de ser necesario. A esto se le llama
\emph{cache snooping}.

La memoria cache fue implementada desde los procesadores Pentium de Intel. A
este tipo de cache se le conoce como \emph{L1-cache}. Modelos recientes
implementan varios niveles de cache (L2, L3, etc) donde cada nivel es mas lento
que el anterior pero mas rápido que la memoria RAM. El manejo de los distintos
niveles es dejado a nivel de hardware por lo que sistemas operativos como Linux
asumen solo un nivel de cache.

Existe una segunda cache usada para acelerar el cálculo de las pysichal address
llamado \emph{Translation Lookaside Buffer} (TLB). Cuando una pysichal address
es calculada, se utilizan la Page Tables la primera vez y el resultado se
guarda en la TLB para que en posteriores accesos se utilize la dirección
almacenada en la TLB en lugar de hacer todo el cálculo.

\subsection{Pyshical Address Real World Example}

En esta última sección se mostrará como Windows 8 en su versión de 64 bits
almacena las linear address, llamadas virtual address en los sistemas
operativos modernos, y hacer el cálculo de la physical address.

Debido a que windows mantiene los valores de las diferentes tablas en registros
especiales nos puede mostrar el contenido y dirección de cada una de las cuatro
tablas usadas. Utilizaremos el contenido asociado que nos muestra el kernel
debugger de la última page table. Se deja como experimento personal al lector
el comprobar las direcciones de las otras tablas. El valor que nos interesa es
\texttt{} que son los bits del 12 al 20 del contenido de la page table, este
valor coincide con el valor pfn mostrado por el kernel debugger. A este valor
de concatenamos el offset de la virtual address que son los 12 bits menos
significativos. Esto nos da \texttt{}.

Para comprobar que nuestro caculo es correcto, le pedimos al kernel debugger
que ns muestre el contenido de la physical address caculada usando la
instrucción \texttt{!dd}.

%%% Local Variables: 
%%% mode: latex
%%% TeX-master: "index"
%%% End: 